\documentclass[a4paper, 12pt]{article}

% PACOTES
\usepackage[utf8]{inputenc} % Caracteres em geral
\usepackage[brazilian]{babel} % Idioma da página
\usepackage[top=2cm, bottom=2cm, left=2cm, right=2cm]{geometry} % margens
\usepackage{fancyhdr} % Cabeçalho
\usepackage{ulem} % Destacar, sublinhar, riscar e etc...
\usepackage{graphicx} % Figuras
\usepackage{booktabs} % Tabelas
\usepackage{multicol} % Multicolunas para tabelas
\usepackage{multirow} % Multilinhas para tabelas
\usepackage{colortbl} % Colorir tabelas
\usepackage[table]{xcolor} % Para usar !25 !50 !75 nas cores
\usepackage{float} % Para manter as tabelas no canto certo
\usepackage{amsmath} % Matrizes
\usepackage{hyperref} % Hyperlinks
\usepackage{tikz} % TikZ - Desenhos e Gráficos
\usepackage{amsmath,amssymb} % Mais símbolos matemáticos
\usepackage[shortlabels]{enumitem} % Usar letras no enumerate

% PREÂMBULOS
\pagenumbering{Roman}
\tikzstyle{every picture}+=[remember picture,inner xsep=0,inner ysep=0.25ex]
\newcommand{\espaco}{\vspace{10pt}}

% INÍCIO DO DOCUMENTO
\begin{document}

\section{Definição Informal de Limite de uma
Função}
\subsection{Exercícios}

\begin{enumerate}
    \item Use a simplificação algébrica para achar o limite, caso exista, utilizando valores próximos
    \begin{enumerate}[(a)]

        \item $\lim_{x \to -3} \frac{(x + 3) \cdot (x-4)}{(x+1) \cdot (x+3)}$
        
        \begin{align*}
            \lim_{x \to -3} \frac{(x + 3) \cdot (x-4)}{(x+1) \cdot (x+3)} &= \lim_{x \to -3} \frac{x-4}{x+1} \\
            &= \frac{\lim_{x \to -3}(x-4)}{\lim_{x \to -3} (x+1)} \\
            &= \frac{\lim_{x \to -3}x-\lim_{x \to -3}4}{\lim_{x \to -3} x+\lim_{x \to -3}1} \\
            &= \frac{-3 -4}{-4 + 1} \\
            &= \frac{7}{3}
        \end{align*}

        \item $\lim_{x \to -1} \frac{(x+1) \cdot (x^{2} + 3)}{x+1}$
        
        \begin{align*}
            \lim_{x \to -1} \frac{(x+1) \cdot (x^{2} + 3)}{x+1} &= \lim_{x \to -1}(x^2 + 3) \\
            &= (-1)^2 + 3 \\
            & = 4
        \end{align*}
        
        \item $\lim_{x \to 2} \frac{x^2 - 4}{x - 2}$
        
        \begin{align*}
            \lim_{x \to 2} \frac{x^2 - 4}{x - 2} &= \lim_{x \to 2} \frac{x^2 - 2^2}{x - 2} \\
            &= \lim_{x \to 2} \frac{(x+2) \cdot (x-2)}{x - 2} \\
            &= \lim_{x \to 2}(x+2)\\
            &= 2 + 2 \\
            &= 4
        \end{align*}

        \item  $\lim_{x \to 0} \frac{(x+h)^2 - -^2}{h}$
        
        \begin{align*}
            \lim_{h \to 0} \frac{(x+h)^2 - -^2}{h} &= \lim_{h \to 0} \frac{x^2 + 2hx + h^2 -x^2}{h}\\
            &= \lim_{h \to 0} \frac{h^2 + 2hx}{h}\\
            &= \lim_{h \to 0} \frac{h\cdot (h + 2x)}{h}\\
            &= \lim_{h \to 0} (h + 2x)\\
            &= 2x
        \end{align*}

        \item $\lim_{z \to 4} \frac{z-4}{z^2 - 2z - 8}$
        
        \begin{align*}
            \lim_{z \to 4} \frac{z-4}{z^2 - 2z - 8} &= \lim_{z \to 4} \frac{z-4}{(z+2) \cdot (z-4)}\\
            &= \lim_{z \to 4} \frac{1}{z+2}\\
            &= \frac{\lim_{z \to 4}1}{\lim_{z \to 4}(z+2)} \\
            &= \frac{1}{6}
        \end{align*}
        
        \item $\lim_{x \to 3} \frac{2x^3 - 6x^2 + x - 3}{x-3}$
        
        \begin{align*}
            \lim_{x \to 3} \frac{2x^3 - 6x^2 + x - 3}{x-3} &=
            \lim_{x \to 3} \frac{(x-3) \cdot (2x^2 +1)}{x-3}\\
            &= \lim_{x \to 3} (2x^2 +1) \\
            &= 2 \cdot 3^2 + 1 \\
            &= 19
        \end{align*}

        \item $\lim_{r \to 1} \frac{r^2 - r}{r2 + r - 2}$
        
        \begin{align*}
            \lim_{r \to 1} \frac{r^2 - r}{r2 + r - 2} &=
            \lim_{r \to 1} \frac{r \cdot (r - 1)}{(r-1) \cdot (r + 2)}\\
            &= \lim_{r \to 1} \frac{r}{r + 2} \\
            &= \frac{\lim_{r \to 1}r}{\lim_{r \to 1}(r + 2)}\\
            &= \frac{1}{3}
        \end{align*}

        \item $\lim_{r \to -3} \frac{r^2 + 2r - 3}{r^2 + 7r + 12}$
        
        \begin{align*}
            \lim_{r \to -3} \frac{r^2 + 2r - 3}{r^2 + 7r + 12} &= \lim_{r \to -3} \frac{(r-1) \cdot (r+3)}{(r+3) \cdot (r+4)} \\
            &= \lim_{x \to -3} \frac{r-1}{r+4} \\
            &= \frac{\lim_{x \to -3}(r-1)}{\lim_{x \to -3}(r+4)} \\
            &= -4
        \end{align*}

        \item $\lim_{h \to -2} \frac{h^3 +8}{h+2}$
        
        \begin{align*}
            \lim_{h \to -2} \frac{h^3 +8}{h+2} &= \lim_{h \to -2} \frac{(h+2) \cdot (h^2 -2h +4)}{h+2} \\
            &= \lim_{h \to -2} (h^2 -2h +4) \\
            &= (-2)^2 - 2 \cdot (-2) + 4 \\
            &= 12
        \end{align*}

        \item $\lim_{h \to -2} \frac{h^3 -8}{h-2}$
        
        \begin{align*}
            \lim_{h \to -2} \frac{h^3 -8}{h-2} &= \lim_{h \to -2} \frac{(h-2) \cdot (h^2 +2h +4)}{h-2} \\
            &= \lim_{h \to -2} (h^2 +2h +4) \\
            &= (-2)^2 + 2 \cdot (-2) + 4 \\
            &= 4
        \end{align*}
        
    \end{enumerate}
\end{enumerate}


\section{Propriedades dos limites}
\subsection{Exercícios}
\begin{enumerate}
    \item Calcule o limite justificando cada passagem com as propriedades dos limites que foram usadas.
    
    \begin{enumerate}
        \item $\lim_{x \to 4}(5x^2 - 2x + 3)$
        
        \begin{align*}
            \lim_{x \to 4}(5x^2 - 2x + 3) &= 5 \cdot 4^2 - 2 \cdot 4 + 3 \tag*{[l. pol.]}\\
            &= 75
        \end{align*}

        \item $\lim_{x \to 8} (1 + \sqrt[3]{x})(2-6x^2 + x^3)$
        
        \begin{align*}
            \lim_{x \to 8} (1 + \sqrt[3]{x})(2-6x^2 + x^3) &= \lim_{x \to 8}(1 + \sqrt[3]{x}) \cdot \lim_{x \to 8}(2-6x^2 + x^3) \tag*{[l. prod.]}\\
            &= (1 + \sqrt[3]{8}) (2-6 \cdot 8^2 + 8^3) \tag*{[l. pol.]}\\
            &= (1 + 2) (2-384 + 512) \\
            &= 390
        \end{align*}

        \item $\lim_{x \to 1} \left (\frac{1 + 3x}{1 + 4x^2 + 3x^3} \right)^3$
        
        \begin{align*}
            \lim_{x \to 1} \left (\frac{1 + 3x}{1 + 4x^2 + 3x^3} \right)^3 &= \left (\lim_{x \to 1} \frac{1 + 3x}{1 + 4x^2 + 3x^3} \right)^3 \tag*{[l. pot.]}\\
            &= \left (\frac{\lim_{x \to 1}(1 + 3x)}{\lim_{x \to 1}(1 + 4x^2 + 3x^3)} \right)^3 \tag*{[l. quo.]}\\
            &= \left (\frac{(1 + 3)}{(1 + 4 + 3)} \right)^3 \tag*{[l. pol.]}\\
            &= \left( \frac{4}{8} \right)^3\\
            &= \frac{1}{8}
        \end{align*}

        \item $\lim_{x \to -1} \frac{x-2}{x^2 + 4x - 3}$
        
        \begin{align*}
            \lim_{x \to -1}\frac{x-2}{x^2 + 4x - 3} &= \frac{\lim_{x \to -1}(x-2)}{\lim_{x \to -1}(x^2 + 4x - 3)} \tag*{[l. quo.]}\\
            &= \frac{-1 -2}{(-1)^2 + 4(-1) - 3} \tag*{[l. pol.]} \\
            &= \frac{-3}{-6} \\
            &= \frac{1}{2}
        \end{align*}

        \item $\lim_{t \to -1} (t^2 + 1)^3 (t+ 3)^5$
        
        \begin{align*}
            \lim_{t \to -1} (t^2 + 1)^3 (t+ 3)^5 &= \lim_{t \to -1}(t^2 + 1)^3 \lim_{t \to -1} (t+ 3)^5 \tag*{[l. prod.]}\\
            &= \left(\lim_{t \to -1}(t^2 + 1)\right)^3 \left(\lim_{t \to -1}(t+ 3) \right)^5 \tag*{[l. pot.]}\\
            &= ((-1)^2 + 1)^3 (-1 + 3) ^5 \tag*{[l. pol.]}\\
            &= 2^3 \cdot 2^5 \\
            &= 256
        \end{align*}

        \item $\lim_{u \to -2} \sqrt{u^4 + 3u + 6}$
        
        \begin{align*}
            \lim_{u \to -2} \sqrt{u^4 + 3u + 6} &= \sqrt{\lim_{u \to -2}(u^4 + 3u + 6)}\tag*{[l. raíz.]}\\
            &= \sqrt{(-2)^4 + 3(-2) + 6} \tag*{[l. pol.]}\\
            &= \sqrt{16 - 6 + 6} \\
            &= 4
        \end{align*}

        \item $\lim_{x \to 2} \sqrt[3]{x^2 - 4x^2 + 5x - x}$
        
        \begin{align*}
            \lim_{x \to 2} \sqrt[3]{x^3 - 4x^2 + 5x - x} &= \sqrt[3]{\lim_{x \to 2}(x^3 - 4x^2 + 5x - x)}\tag*{[l. raíz.]} \\
            &= \sqrt[3]{2^3 - 4\cdot 2^2 + 5\cdot 2 - x} \tag*{[l. pol.]}\\
            &= \sqrt[3]{8 - 16 + 10 - 2}\\
            &= 0
        \end{align*}

        \item $\lim_{x \to -1} \frac{x^2 + 3x - 5x^5}{x-1}$ 
        
        \begin{align*}
            \lim_{x \to -1} \frac{x^2 + 3x - 5x^5}{x-1} &= \frac{\lim_{x \to -1}(x^2 + 3x - 5x^5)}{\lim_{x \to -1}(x-1)} \tag*{[l. quo.]}\\
            &= \frac{(-1)^2 + 3(-1) - 5(-1)^5}{-1-1} \tag*{[l. pol.]}\\
            &= - \frac{3}{2}
        \end{align*}

        \item $\lim_{x \to -2} \left( \frac{x^5 - x^4 + x^3 - x^2}{2-x}^2 \right)$
        
        \begin{align*}
            \lim_{x \to -2} \left( \frac{x^5 - x^4 + x^3 - x^2}{2-x}^2 \right) &= \left( \lim_{x \to -2} \left(\frac{x^5 - x^4 + x^3 - x^2}{2-x} \right)\right)^2 \tag*{[l. pot.]}\\
            &=\left( \frac{\lim_{x \to -2}(x^5 - x^4 + x^3 - x^2)}{\lim_{x \to -2}(2-x)} \right)^2 \tag*{[l. quo.]} \\
            &=\left( \frac{(-2)^5 - (-2)^4 + (-2)^3 - (-2)^2}{2-(-2)} \right)^2 \tag*{[l. pol.]}\\
            &=\left( \frac{-60}{4} \right)^2 \\
            &=255
        \end{align*}

        \item $\lim_{x \to -1} \sqrt[4]{2x^4 - x^3 + 3x^2 + 1}$
        
        \begin{align*}
            \lim_{x \to -1} \sqrt[4]{2x^4 - x^3 + 3x^2 + 1} &= \sqrt[4]{\lim_{x \to -1}(2x^4 - x^3 + 3x^2 + 1)}  \tag*{[l. raíz.]}\\
            &= \sqrt[4]{2(-1)^4 - (-1)^3 + 3(-1)^2 + 1} \tag*{[l. pol.]}\\
            &= \sqrt[4]{2 +1 + 3 + 1} \\
            &= \sqrt[4]{7}
        \end{align*}
    \end{enumerate}

    \item Calcule o limite, caso existir.
    
    \begin{enumerate}
        \item $\lim_{x \to 4} 7$
        
        \begin{align*}
            \lim_{x \to 4} 7 &= 7 \tag*{[l. const.]}
        \end{align*}

        \item $\lim_{x \to -1} \frac{2}{3}$
        
        \begin{align*}
            \lim_{x \to -1} \frac{2}{3} &= \frac{2}{3} \tag*{[l. const.]}
        \end{align*}

        \item $\lim_{x \to 2}(5x^3 + x)$
        
        \begin{align*}
            \lim_{x \to 2}(5x^3 + x) &= 5\cdot 2^3 + 2 \tag*{[l. pol.]} \\
            &= 42
        \end{align*}

        \item $\lim_{x \to -4} \left( 4x^2 - \frac{1}{2}x \right)$
        
        \begin{align*}
            \lim_{x \to -4} \left( 4x^2 - \frac{1}{2}x \right) &= 4(-4)^2 - \frac{1}{2}(-4) \tag*{[l. pol.]}\\
            &= 66
        \end{align*}

        \item $\lim_{x \to 3} (3x^2 + x - 1)$
        
        \begin{align*}
            \lim_{x \to 3} (3x^2 + x - 1) &= 3\cdot 3^2 + 3 - 1 \tag*{[l. pol.]}\\
            &= 29
        \end{align*}

        \item $\lim_{x \to 0} (x^4 -x^3 + x^2 +1)$
        
        \begin{align*}
            \lim_{x \to 0} (x^4 -x^3 + x^2 +1) &= 0 - 0 +0 + 1 \tag*{[l. pol.]}\\
            &= 1
        \end{align*}

        \item $\lim_{x \to 1} 6x^2$
        
        \begin{align*}
            lim_{x \to 1} 6x^2 = 6 \tag*{[l. pol.]}
        \end{align*}

        \item $\lim_{x \to 3} (x-1) (4-x)$
        
        \begin{align*}
            \lim_{x \to 3} (x-1) (4-x) &= \lim_{x \to 3}(x-1) \lim_{x \to 3}(4-x) \tag*{[l. prod.]}\\
            &= (3-1) (4-3) \tag*{[l. pol.]}\\
            &= 2
        \end{align*}

        \item $\lim_{x \to 3} \frac{4x^2}{x+1}$
        
        \begin{align*}
            \lim_{x \to 3} \frac{4x^2}{x+1} &= \frac{\lim_{x \to 3}(4x^2)}{\lim_{x \to 3}(x+1)} \tag*{[l. quo]}\\
            &= \frac{4\cdot 3^2}{3+1} \tag*{[l. pol.]} \\
            &= \frac{36}{4}\\
            &= 9
        \end{align*}

        \item $\lim_{x \to 5}\frac{(x^3)}{x^2 -1}$
        
        \begin{align*}
            \lim_{x \to 5}\frac{x^3}{x^2 -1} &= \frac{\lim_{x \to 5}x^3}{\lim_{x \to 5}(x^2 -1)} \tag*{[l. quo.]}\\
            &= \frac{5^3}{5^2 -1} \tag*{[l. pol.]}\\
            &= \frac{125}{24}
        \end{align*}

        \item $\lim_{x \to -1}(2x -1)^6$
        
        \begin{align*}
            \lim_{x \to -1}(2x -1)^6 &= \left( \lim_{x \to -1}(2x -1) \right)^6 \tag*{[l. pot.]}\\
            &= (2\cdot (-1) -1)^6 \tag*{[l. pol.]}\\
            &= (-3)^6\\
            &= 729
        \end{align*}

        \item $\lim_{x \to 2} (3x^3 - 2x^2 + 5x - 1)^2$
        
        \begin{align*}
            \lim_{x \to 2} (3x^3 - 2x^2 + 5x - 1)^2 &= \left(  \lim_{x \to 2}(3x^3 - 2x^2 + 5x - 1) \right)^2 \tag*{[l. pot.]}\\
            &= (3\cdot 2^3 - 2\cdot 2^2 + 5\cdot 2 - 1)^2 \tag*{[l. pol.]} \\
            &= (25)^2 \\
            &= 625
        \end{align*}

        \item $\lim_{x \to 1} \sqrt[4]{81x^4}$
        
        \begin{align*}
            \lim_{x \to 1} \sqrt[4]{81x^4} &= \lim_{x \to 1} \sqrt[4]{3^4x^4}\\
            &= \lim_{x \to 1} 3x \tag*{[l. pol.]}\\
            &= 3
        \end{align*}

        \item $\lim_{x \to 4} \sqrt[3]{x^2}$
        
        \begin{align*}
            \lim_{x \to 4} \sqrt[3]{x^2} &= \sqrt[3]{\lim_{x \to 4}x^2} \tag*{[l. raíz]}\\
            &= \sqrt[3]{16} \tag*{[l. pol.]}
        \end{align*}

        \item $\lim_{x \to \sqrt[]{2}} (x^2 + 3) (x-2)$
        
        \begin{align*}
            \lim_{x \to \sqrt[]{2}} (x^2 + 3) (x-2) &= \lim_{x \to \sqrt[]{2}}(x^2 + 3) \lim_{x \to \sqrt[]{2}}(x-2) \tag*{[l. prod.]} \\
            &= (\sqrt[]{2}^2 + 3) (\sqrt[]{2}-2) \tag*{[l. pol.]} \\
            &= 5 \cdot (\sqrt[]{2} -2) \\
            &= 5\sqrt[]{2} - 10
        \end{align*}

        \item $\lim_{x \to \frac{1}{2}} (4x-1)^50$
        
        \begin{align*}
            \lim_{x \to \frac{1}{2}} (4x-1)^{50} &= \left( \lim_{x \to \frac{1}{2}}(4x-1) \right)^{50} \tag*{[l. pot.]}\\
            &= 1^{50} \tag*{[l. pol.]}\\
            &= 1
        \end{align*}

        \item $\lim_{x \to 16} \frac{2\sqrt{x} + x^{\frac{3}{2}}}{\sqrt[4]{x} + 4}$
        
        \begin{align*}
            \lim_{x \to 16} \frac{2\sqrt[]{x} + x^{\frac{3}{2}}}{\sqrt[4]{x} + 4} &= \frac{\lim_{x \to 16}(2\sqrt[]{x} + x^{\frac{3}{2}})}{\lim_{x \to 16}(\sqrt[4]{x} + 4)} \tag*{[l. quo.]}\\
            &= \frac{2\sqrt{16} + 16^{\frac{3}{2}}}{\sqrt[4]{16} + 4} \tag*{[l. pol.]}\\
            &= 12
        \end{align*}

        \item $\lim_{x \to 1} \left( \sqrt{x} + \frac{1}{\sqrt{x}} \right)^6$
        
        \begin{align*}
            \lim_{x \to 1} \left( \sqrt{x} + \frac{1}{\sqrt{x}} \right)^6 &= \left( \lim_{x \to 1}\left(\sqrt{x} + \frac{1}{\sqrt{x}}\right) \right)^6 \tag*{[l. pot.]}\\
            &= \left( \lim_{x \to 1}\sqrt{x} + \lim_{x \to 1}\left(\frac{1}{\sqrt{x}}\right) \right)^6 \tag*{[l. soma.]} \\
            &= \left( \lim_{x \to 1}\sqrt{x} + \frac{1}{\lim_{x \to 1}\sqrt{x}}\right)^6 \tag*{[l. quo.]} \\
            &= \left( \sqrt{\lim_{x \to 1}x} + \frac{1}{\sqrt{\lim_{x \to 1} x}}\right)^6 \tag*{[l. raíz.]} \\
            &= \left( \sqrt{1} + \frac{1}{\sqrt{1}} \right) ^6 \tag*{[l. pol.]}\\
            &= 2^6 \\
            &= 64
        \end{align*}

        \item $\lim_{x \to 4} \sqrt[3]{x^2 - 5x - 4}$
        
        \begin{align*}
            \lim_{x \to 4} \sqrt[3]{x^2 - 5x - 4} &= \sqrt[3]{\lim_{x \to 4}(x^2 - 5x - 4)} \tag*{[l. raíz]}\\
            &= \sqrt[3]{4^2 - 5\cdot 4 - 4} \tag*{[l. pol.]}\\
            &= \sqrt[3]{-8}\\
            &= -2
        \end{align*}

        \item $\lim_{x \to 3} \frac{\sqrt[3]{2 + 5x - 3x^2}}{x^2 -1}$
        
        \begin{align*}
            \lim_{x \to 3} \frac{\sqrt[3]{2 + 5x - 3x^2}}{x^2 -1} &= \frac{\lim_{x \to 3}\sqrt[3]{2 + 5x - 3x^2}}{\lim_{x \to 3}(x^2 -1)} \tag*{[l. quo.]}\\
            &= \frac{\sqrt[3]{\lim_{x \to 3}(2 + 5x - 3x^2)}}{\lim_{x \to 3}(x^2 -1)} \tag*{[l. raíz.]}\\
            &= \frac{\sqrt[3]{2 + 5\cdot 3 - 3\cdot 3^2}}{3^2 -1} \tag*{[l. pol.]} \\
            &= \frac{\sqrt[3]{-64}}{8} \\
            &= -\frac{1}{2}
        \end{align*}

        \item $\lim_{x \to 4}(3x-4)$
        
        \begin{align*}
            \lim_{x \to 4}(3x-4) &= 3\cdot 4 -4 \tag*{[l. pol.]}\\
            &= 8
        \end{align*}

        \item $\lim_{x \to -2}(-3x+1)$
        
        \begin{align*}
            \lim_{x \to -2}(-3x+1) &= -3 \cdot (-2) +1 \tag*{[l. pol.]} \\
            &= 7
        \end{align*}

        \item $\lim_{x \to -2} \frac{x-5}{4x+3}$
        
        \begin{align*}
            \lim_{x \to -2} \frac{x-5}{4x+3} &= \frac{\lim_{x \to -2}(x-5)}{\lim_{x \to -2}(4x+3)} \tag*{[l. quo.]}\\
            &= \frac{-2-5}{4\cdot (-2)+3} \tag*{l. pol.} \\
            &= \frac{7}{5}
        \end{align*}

        \item $\lim_{x \to 4} \frac{2x -1}{3x+1}$
        
        \begin{align*}
            \lim_{x \to 4} \frac{2x -1}{3x+1} &= \frac{\lim_{x \to 4}(2x -1)}{\lim_{x \to 4}(3x+1)} \tag*{[l. quo.]}\\
            &= \frac{2\cdot 4 -1}{3\cdot 4+1} \tag*{[l. pol.]}\\
            &= \frac{7}{13}
        \end{align*}

        \item $\lim_{x \to 1} (-2x +5)^4$
        
        \begin{align*}
            \lim_{x \to 1} (-2x +5)^4 &= \left( \lim_{x \to 1} (-2x +5)  \right)^4 \tag*{[l. pot.]}\\
            &= (-2 +5)^4 \tag*{[l. pol.]}\\
            &= 3^4 \\
            &= 81
        \end{align*}

        \item $\lim_{x \to -2} (3x - 1)^5$
        
        \begin{align*}
            \lim_{x \to -2} (3x - 1)^5 &= \left( \lim_{x \to -2} (3x - 1) \right)^5 \tag*{[l. pot.]}\\
            &= (3 \cdot (-2) - 1)^5 \tag*{[l. pol.]}\\
            &= (-7)^5\\
            &= -16.807
        \end{align*}
    \end{enumerate}
\end{enumerate}

\newpage
\section{Limites Infinitos}
\subsection{Exercícios}

\begin{enumerate}
    \item Seja $f(x)$ a função definida pelo gráfico:
    
    \begin{figure}[htbp]
        \centering
        \begin{tikzpicture}
            % Eixos
            \draw[->] (-1.5,0) -- (6,0) node[right] {$x$};
            \draw[->] (0,-1.5) -- (0,4) node[above] {$y$};

            % Legenda
            \draw[](-0.2,-1.3) -- (-0.2,-1.3) node[left] {$-1$};
            \draw[](-0.2,1) -- (-0.2,1) node[left] {$1$};
            \draw[](-0.2,3) -- (-0.2,3) node[left] {$3$};
            \draw[](3,-0.2) -- (3,-0.2) node[below] {$3$};

            % Gráfico
            \draw[line width=2pt](-1,-1) -- (2.9,-1) node[xshift= 2pt]{$\circ$};
            \draw[line width=2pt](6,3) -- (3.1,3) node[xshift=-2pt]{$\circ$};
            \draw[](3,1) -- (3,1) node[]{$\bullet$};

            % Linhas de referência
            \draw[loosely dotted](0,3) -- (2.9,3);
            \draw[loosely dotted](0,1) -- (3,1);
            
        \end{tikzpicture}
        \caption{$f(x)$}
    \end{figure}

    Intuitivamente, encontre se existir:
    \begin{enumerate}
        \item $\lim_{x \to 3^{-}}f(x)$ = -1
        \item $\lim_{x \to 3^{+}} f(x)$ = 3
        \item $\lim_{x \to 3} f(x)$ = Não existe, pois os limites laterais são diferentes.
        \item $\lim_{x \to -\infty}f(x)$ = -1
        \item $\lim_{x \to +\infty} f(x)$ = 3
        \item $\lim_{x \to \infty} f(x)$ = Não existe, pois os limites laterais são diferentes.
        \item $\lim_{x \to 4} f(x)$ = 3
    \end{enumerate}

    \newpage
    \item Seja $f(x)$ a função definida pelo gráfico:
    
    \begin{figure}[htbp]
        \centering
        \begin{tikzpicture}
            % Eixos
            \draw[->] (-4,0) -- (2,0) node[right] {$x$};
            \draw[->] (0,-1) -- (0,4) node[above] {$y$};

            % Legenda
            \draw[](0.2,2) -- (0.2,2) node[right] {$2$};
            \draw[](-2,-0.2) -- (-2,-0.2) node[below] {$-2$};

            % Gráfico
            \draw[line width=2pt](-4,2) -- (-2,0);
            \draw[line width=2pt](-2,0) -- (2,4);
            \draw[fill=white] (-2,0) circle (2pt);

        \end{tikzpicture}
        \caption{$f(x)$}
    \end{figure}

    \begin{enumerate}
        \item $\lim_{x \to 2^{-}} f(x)$ = -2
        \item $\lim_{x \to 2^{+}} f(x)$ = -2
        \item $\lim_{x \to 2} f(x)$ = -2
        \item $\lim_{x \to -\infty} f(x)$ = $\infty$
        \item $\lim_{x \to +\infty} f(x)$ = $\infty$
        \item $\lim_{x \to \infty} f(x)$ = $\infty$
    \end{enumerate}

    \vspace{1cm}

    \item Seja $f(x)$ a função definida pelo gráfico:
    
    \begin{figure}[htbp]
        \centering
        \begin{tikzpicture}
            % Eixos
            \draw[->] (-0.5,0) -- (4.5,0) node[right] {$x$};
            \draw[->] (0,-0.5) -- (0,3) node[above] {$y$};

            % Legenda
            \draw[](-0.2,0.5) -- (-0.2,0.5) node[left] {$1/2$};
            \draw[](1,-0.2) -- (1,-0.2) node[below] {$1$};

            % Gráfico
            \draw[line width=2pt](-0.5,-0.5) -- (1,0.5) node{$\bullet$};
            \draw[line width=2pt, rotate=180] (-3.1,-0.6) arc (90:0:2);

            % Linhas de referência
            \draw[loosely dotted](1,3) -- (1,-0.5);
            \draw[loosely dotted](4.5,0.5) -- (0,0.5);

        \end{tikzpicture}
        \caption{$f(x)$}
    \end{figure}

    \begin{enumerate}
        \item $\lim_{x \to 1^{-}}$ = 0.5
        \item $\lim_{x \to 1^{+}}$ = $+\infty$
        \item $\lim_{x \to -\infty}$ = $-\infty$
        \item $\lim_{x \to +\infty}$ = 0.5
        \item $\lim_{x \to 1}$ = não existe, pois os limites laterais não são iguais.
    \end{enumerate}

    \newpage

    \item Calcule os limites laterais.
    
    \begin{enumerate}
        \item $\lim_{x \to 1^{+}} \frac{2x}{x-1}$ = $+ \infty$ 
        
        \espaco
        Veja que, quanto mais o $x$ se aproxima de 1 pela direta, maior será $f(x)$. Sendo assim, podemos assumir que o limite desse função quando $x \to 1^{+}$ é $+ \infty$
        \espaco

        \item $\lim_{x \to 2^{-}} \frac{x^2}{x-2}$ = $- \infty$
        \item $\lim_{x \to 3^{+}} \frac{x^2 + 5x + 1}{x^2 - 2x - 3}$ = $+ \infty$
        \item $\lim_{x \to 5^{-}} \frac{\sqrt{25 - x^2}}{x-5} = - \infty$
        \item $\lim_{x \to 2^+} \frac{x^3}{x-2} = - \infty$
        \item $\lim_{x \to 3^+} \frac{x^2 - 5x + 1}{x^2 -2x + 3}$
        
        \begin{align*}
            \lim_{x \to 3^+} \frac{x^2 - 5x + 1}{x^2 -2x + 3} &= \frac{\lim_{x \to 3^+}(x^2 - 5x + 1)}{\lim_{x \to 3^+}(x^2 -2x + 3)} \tag*{[l. quo.]}\\
            &= \frac{3^2 - 5 \cdot 3 + 1}{3^2 -2 \cdot 3 + 3} \tag*{[l. pol.]}\\
            &= \frac{-5}{6}
        \end{align*}

        \item $\lim_{h \to 0^+} \frac{(3+h)^2 - 9}{h}$
        
        \begin{align*}
            \lim_{h \to 0^+} \frac{(3+h)^2 - 9}{h} &= \lim_{h \to 0^+} \frac{h^2 +6h +9 - 9}{h}\\
            &=\lim_{h \to 0^+} \frac{(3+h)^2 - 9}{h}\\
            &= \lim_{h \to 0^+} \frac{h^2 +6h}{h}\\
            &= \lim_{h \to 0^+} \frac{h \cdot (h + 6)}{h}\\
            &= \lim_{h \to 0^+} (h+6)\\
            &= 6 \tag*{[l. pol.]}
        \end{align*}

        \item $\lim_{x \to 1^-} \frac{x^3 -1}{x^2 -1}$
        
        \begin{align*}
            \lim_{x \to 1^-} \frac{x^3 -1}{x^2 -1} &= \lim_{x \to 1^-} \frac{(x-1) \cdot (x^2 + x + 1)}{(x-1) \cdot (x+1)}\\
            &= \lim_{x \to 1^-} \frac{x^2 + x + 1}{x+1}\\
            &= \frac{\lim_{x \to 1^-}(x^2 + x + 1)}{\lim_{x \to 1^-}(x+1)} \tag*{[l. quo.]}\\
            &= \frac{1^2 + 1 + 1}{1+1} \tag*{[l. pol.]}\\
            &= \frac{3}{2}
        \end{align*}

        \item $\lim_{x \to 2^+} \frac{x^2 - 4}{2-x}$
        
        \begin{align*}
            \lim_{x \to 2^+} \frac{x^2 - 4}{2-x} &= \lim_{x \to 2^+} \frac{(x+2)(x-2)}{-1 \cdot (x-2)}\\
            &= \lim_{x \to 2^+} \frac{(x+2)}{-1}\\
            &= \lim_{x \to 2^+}-(x+2) \\
            &= -4 \tag*{[l. pol.]}
        \end{align*}

        \item $\lim_{x \to 0^+}\frac{\sqrt{x + 2} - \sqrt{2}}{x}$
        
        \begin{align*}
            \lim_{x \to 0^+}\frac{\sqrt{x + 2} - \sqrt{2}}{x} &= \lim_{x \to 0^+}\frac{\sqrt{x + 2} - \sqrt{2}}{x} \cdot \frac{\sqrt{x + 2} + \sqrt{2}}{\sqrt{x + 2} + \sqrt{2}}\\
            &= \lim_{x \to 0^+}\frac{x}{x\cdot \sqrt{x + 2} + \sqrt{2}}\\
            &= \lim_{x \to 0^+}\frac{1}{\sqrt{x + 2} + \sqrt{2}}\\
            &= \frac{1}{\sqrt{2} + \sqrt{2}} \tag*{[l. pol.]}\\
            &= \frac{1}{2\sqrt{2}}\\
            &= \frac{\sqrt{2}}{4}
        \end{align*}

    \end{enumerate}
    
    \item Para cada função $f(x)$ abaixo, calcule $\lim_{x \to a^+} f(x)$, $\lim_{x \to a^+} f(x)$ e $\lim_{x \to a} f(x)$, quando existirem.
    
    \newcommand{\naoexiste}{$\rightarrow$ não existe, pois os limites laterais são diferentes.}
    
    \begin{enumerate}
        \item $f(x) = \frac{4}{x-6}$, $a=6$
        
        \begin{enumerate}
            \item $\lim_{x \to 6^+} \frac{4}{x-6} = + \infty$
            \item $\lim_{x \to 6^-} \frac{4}{x-6} = - \infty$
            \item $\lim_{x \to 6} \frac{4}{x-6}$ \naoexiste
        \end{enumerate}

        \item $f(x) = \frac{3}{1-x}$, $a=1$
        
        \begin{enumerate}
            \item $\lim_{x \to 1^+} \frac{3}{1-x} = - \infty$
            \item $\lim_{x \to 1^-} \frac{3}{1-x} = + \infty$
            \item $\lim_{x \to 1} \frac{3}{1-x}$ \naoexiste
        \end{enumerate}

        \item $f(x) = \frac{x+5}{x}$, $a=0$
        
        \begin{enumerate}
            \item $\lim_{x \to 0^+} \frac{x+5}{x} = + \infty$
            \item $\lim_{x \to 0^-} \frac{x+5}{x} = - \infty$
            \item $\lim_{x \to 0} \frac{x+5}{x}$ \naoexiste
        \end{enumerate}

        \item $f(x) = \frac{x}{2-x}$, $a=2$
        
        \begin{enumerate}
            \item $\lim_{x \to 2^+} \frac{x}{2-x} = - \infty$
            \item $\lim_{x \to 2^-} \frac{x}{2-x} = + \infty$
            \item $\lim_{x \to 2} \frac{x}{2-x}$ \naoexiste
        \end{enumerate}

        \item $f(x) = \frac{x^2}{x-1}$, $a=1$
        
        \begin{enumerate}
            \item $\lim_{x \to 1^+} \frac{x^2}{x-1} = + \infty$
            \item $\lim_{x \to 1^-} \frac{x^2}{x-1} = - \infty$
            \item $\lim_{x \to 1} \frac{x^2}{x-1}$ \naoexiste
        \end{enumerate}

        \item $f(x) = \frac{1}{x}$, $a=0$
        
        \begin{enumerate}
            \item $\lim_{x \to 0^+} \frac{1}{x} = + \infty$
            \item $\lim_{x \to 0^-} \frac{1}{x} = - \infty$
            \item $\lim_{x \to 0} \frac{1}{x}$ \naoexiste
        \end{enumerate}

        \item $f(x) = \frac{1}{x^2}$, $a=0$
        
        \begin{enumerate}
            \item $\lim_{x \to 0^+}\frac{1}{x^2} = + \infty$
            \item $\lim_{x \to 0^-} \frac{1}{x} = + \infty$
            \item $\lim_{x \to 0} \frac{1}{x} = + \infty$
        \end{enumerate}

        \item $f(x) = \frac{-1}{x^2}$, $a=0$
        
        \begin{enumerate}
            \item $\lim_{x \to 0^+}\frac{-1}{x^2} = - \infty$
            \item $\lim_{x \to 0^-} \frac{-1}{x} = - \infty$
            \item $\lim_{x \to 0} \frac{-1}{x} = - \infty$
        \end{enumerate}

        \item $f(x)= \frac{1}{x^3}$, $a=0$
        
        \begin{enumerate}
            \item $\lim_{x \to 0^+} \frac{1}{x^2} = + \infty$
            \item $\lim_{x \to 0^-} \frac{1}{x^2} = - \infty$
            \item $\lim_{x \to 0} \frac{1}{x^2}$ \naoexiste
        \end{enumerate}

        \item $f(x) = 2x+ \frac{1}{x^2}$, $a=0$
        
        \begin{enumerate}
            \item $\lim_{x \to 0^+}2x+\frac{1}{x^2} = + \infty$
            \item $\lim_{x \to 0^-}2x+\frac{1}{x} = + \infty$
            \item $\lim_{x \to 0}2x+\frac{1}{x} = + \infty$
        \end{enumerate}

        \item $f(x) = 5x+ \frac{3}{x-2}$, $a=2$
        
        \begin{enumerate}
            \item $\lim_{x \to 2^+}5x+\frac{3}{x-2}=+\infty$
            \item $\lim_{x \to 2^-}5x+\frac{3}{x-2}=-\infty$
            \item $\lim_{x \to 2}5x+\frac{3}{x-2}$\naoexiste
        \end{enumerate}

        \item $f(x) = \frac{5x}{(x-1)^2}$, $a=1$
        
        \begin{enumerate}
            \item $\lim_{x \to 1^+}\frac{5x}{(x-1)^2}=+\infty$
            \item $\lim_{x \to 1^-}\frac{5x}{(x-1)^2}=+\infty$
            \item $\lim_{x \to 1}\frac{5x}{(x-1)^2}=+\infty$
        \end{enumerate}

        \item $f(x) = \frac{1}{5x(x-1)^2}$, $a=1$
        
        \begin{enumerate}
            \item $\lim_{x \to 1^+}\frac{1}{5x(x-1)^2}=+\infty$
            \item $\lim_{x \to 1^-}\frac{1}{5x(x-1)^2}=+\infty$
            \item $\lim_{x \to 1}\frac{1}{5x(x-1)^2}=+\infty$ 
        \end{enumerate}

        \item $f(x) = \frac{4x}{(x-3)^2}$, $a=3$
        
        \begin{enumerate}
            \item $\lim_{x \to 3^+}\frac{4x}{(x-3)^2}=+\infty$
            \item $\lim_{x \to 3^-}\frac{4x}{(x-3)^2}=+\infty$
            \item $\lim_{x \to 3}\frac{4x}{(x-3)^2}=+\infty$
        \end{enumerate}

        \item $f(x) = \frac{1}{4x(x-3)^2}$, $a=3$
        
        \begin{enumerate}
            \item $\lim_{x \to 3^+}\frac{1}{4x(x-3)^2}=+\infty$
            \item $\lim_{x \to 3^-}\frac{1}{4x(x-3)^2}=+\infty$
            \item $\lim_{x \to 3}\frac{1}{4x(x-3)^2}=+\infty$ 
        \end{enumerate}
    \end{enumerate}   
    
\end{enumerate}

\section{Continuidade de uma Função}
\subsection{Exercícios}

\begin{enumerate}
    \newcommand{\naodefinida}{$f(x)$ não é definida nesse ponto}

    \item Dada a função $f(x) = \frac{1-x}{x+1}$, diga se $f(x)$ é continua nos pontos:
    
    \begin{enumerate}
        \item $x = 0$
        
        \begin{enumerate}
            \item $f(0) = \frac{1-0}{0+1} = 1$
            \item $\lim_{x \to 0} \frac{1-x}{x+1} = \frac{\lim_{x \to 0}(1-x)}{\lim_{x \to 0}(x+1)} = \frac{1-0}{0+1} = 1$
            \item $f(x)$ é contínua em $x=0$ pois $f(0) = \lim_{x \to 0}f(x)$
        \end{enumerate}

        \item $x=-1$
        
        \begin{enumerate}
            \item $f(-1) = \frac{1-(-1)}{-1+1} = \frac{2}{0}$. \naodefinida
            \item $f(x)$ não é continua nesse ponto pois não é definida nesse ponto
        \end{enumerate}
    \end{enumerate}
\end{enumerate}


\end{document}

% ANOTAÇÕES

% \\ ou \newline -> quebra de linha
% \vspace{1cm} -> espaçamento vertical
% \hspace{1cm} -> espaçamento horizontal
% ~ -> equivalente a um espaço
% caracteres especiais -> \# \$ \& \_ \{ \} \textbackslash \textasciitilde
% LaTeX table generator